\section{Introduction}

\subsection{Concept of Autonomy and Safety}

To make a significant contribution to the crypto economy's freedom 
and independence, a new secure and autonomous stablecoin is required 
to minimize the risks associated with centralization and crypto 
market volatility.

Existing stablecoins are susceptible to the following risks due to their 
strong connection to the traditional financial system, which is unpredictable:

\begin{enumerate}
\item Risks due to centralization:
	\begin{enumerate}
		\item Stable tokens can be banned/frozen, or the address can be blacklisted by authorized actors.
    \item The creators of stablecoins can be influenced using legal or other means.
    \item Some functions of stablecoin tokens might stop working if the project
    team stops supporting the project, such as collateral re-balancing or 
    mint/redeem for centralized stables.
    \item Collateral containing cash/treasuries or other stablecoins that include 
    cash/treasuries can collapse due to various situations in the banking system 
    or financial regulation issues.
  \end{enumerate}
\item Risks due to crypto market unpredictability and volatility:
	\begin{enumerate}
    \item A stablecoin might collapse if its internal business model operates incorrectly. 
    Dependency on the internal business model makes its behavior much more unpredictable, for example:
    \begin{itemize}
      \item For continuous support, a stablecoin project team needs funds. Thus, it needs 
      revenue generation in some form. For USDT/USDC and other stablecoins backed 
      by cash/treasures, it is the interest returned from deposits and treasury bond yields. 
      For DAI and other stablecoins backed by crypto assets, the revenue is fees and commissions 
      from lending and borrowing.
    \end{itemize}
    \item Collateral containing crypto assets can depreciate in value due to a decline 
    in the crypto market. If the crypto market is down significantly, it can unpeg 
    a crypto-backed stablecoin from USD.
  \end{enumerate}
\end{enumerate}



The USSD stablecoin aims to avoid the risks associated with existing stablecoins 
by implementing the following features:
\begin{enumerate}
  \item No methods for freezing, blacklisting, banning or pausing transfers are 
  implemented in the smart-contract. The goal is to make the USSD stablecoin completely unstoppable.
  \item USSD is a non-profit, long-term project that exists autonomously without any 
  connection to physical entities. Only code is used, making it non-biased and secure.
  \item The architecture of the USSD stablecoin is designed to be autonomous. In the 
  long-term, the operation of the USSD stablecoin would not be dependent on any person.
  \item Since the USSD protocol operates independently of any business model, it 
  eliminates the risks associated with the bankruptcy of such models.
  \item USSD contains only cryptoassets with no connection to cash in banks or any 
  other traditional financial instruments in the physical world, even low-risk ones.
  \item USSD's collateral structure is designed to maintain a collateral to capitalization 
  ratio of more than 10x.
  \item The USSD smart contract is deployed on the Ethereum main network because it is 
  the most common and trusted decentralized network that allows the building of a smart contract.
\end{enumerate}

\begin{table}
\caption{Comparison of stablecoin features}
\raggedleft
\begin{tabular}{|>{\hspace{0pt}}m{0.156\linewidth}|>{\hspace{0pt}}m{0.158\linewidth}|>{\hspace{0pt}}m{0.175\linewidth}|>{\hspace{0pt}}m{0.175\linewidth}|>{\hspace{0pt}}m{0.14\linewidth}|>{\hspace{0pt}}m{0.133\linewidth}|} 
\hline
\par{}                                   & \textbf{USDT}                                                                             & \textbf{USDC}                                                             & \textbf{BUSD (USDP)}                                                       & \textbf{DAI}                                                              & \textbf{USSD}                                                     \\ 
\hline
Transparency                             & {\cellcolor[rgb]{1,0.843,0.843}}based on auditor reports\textsuperscript{1} & {\cellcolor[rgb]{1,1,0.843}}by paper proof\textsuperscript{2} & {\cellcolor[rgb]{1,1,0.843}}by paper proof \textsuperscript{3} & {\cellcolor[rgb]{0.867,0.91,0.796}}by code\textsuperscript{4} & {\cellcolor[rgb]{0.867,0.91,0.796}}by code            \\ 
\hline
Source of collateral                     & {\cellcolor[rgb]{1,1,0.843}}short-term US treasuries, cash, etc.                          & {\cellcolor[rgb]{1,1,0.843}}short-term US treasuries, cash in US banks     & {\cellcolor[rgb]{1,1,0.843}}short-term US treasuries, cash in US banks      & {\cellcolor[rgb]{0.867,0.91,0.796}}crypto-backed (including USDC)         & {\cellcolor[rgb]{0.867,0.91,0.796}}crypto-backed (including DAI)  \\ 
\hline
Collateral ratio                        & {\cellcolor[rgb]{1,0.847,0.808}}claimed as 100\%                                             & {\cellcolor[rgb]{1,1,0.843}}100\%                                         & {\cellcolor[rgb]{1,1,0.843}}100\%                                          & {\cellcolor[rgb]{0.867,0.91,0.796}}{\raise.17ex\hbox{$\scriptstyle\sim$}}147\%                 & {\cellcolor[rgb]{0.867,0.91,0.796}}{\raise.17ex\hbox{$\scriptstyle\sim$}}500+\%      \\ 
\hline
Regulatory status                         & {\cellcolor[rgb]{1,1,0.843}}Hong Kong entity                                              & {\cellcolor[rgb]{1,1,0.843}}US entity                                     & {\cellcolor[rgb]{1,0.847,0.808}}several entities                           & {\cellcolor[rgb]{1,1,0.843}}US entity                                     & {\cellcolor[rgb]{0.867,0.91,0.796}}no entity                      \\ 
\hline
Centralization level of token management & {\cellcolor[rgb]{1,0.847,0.808}}team can freeze tokens                                     & {\cellcolor[rgb]{1,0.847,0.808}}team can freeze tokens                     & {\cellcolor[rgb]{1,0.847,0.808}}team can freeze tokens                      & {\cellcolor[rgb]{1,0.847,0.808}}team can freeze tokens                     & {\cellcolor[rgb]{0.867,0.91,0.796}}no-one can freze tokens        \\ 
\hline
Project Management                       & {\cellcolor[rgb]{1,0.847,0.808}}centralized, manual                                       & {\cellcolor[rgb]{1,0.847,0.808}}centralized, manual                       & {\cellcolor[rgb]{1,0.847,0.808}}centralized, manual                        & {\cellcolor[rgb]{1,1,0.843}}DAO managed, manual                           & {\cellcolor[rgb]{0.867,0.91,0.796}}autonomous                     \\ 
\hline
Business model                           & {\cellcolor[rgb]{1,1,0.843}}US treasuries yield                                                  & {\cellcolor[rgb]{1,1,0.843}}US treasuries yield                                  & {\cellcolor[rgb]{1,1,0.843}}US treasuries yield                                   & {\cellcolor[rgb]{1,1,0.843}}collateral service fees                         & {\cellcolor[rgb]{0.867,0.91,0.796}}non-profit                     \\
\hline
\end{tabular}
\end{table}

\nopagebreak

\addtocounter{footnote}{1}
\footnotetext{https://tether.to/en/transparency/\#reports}
\addtocounter{footnote}{1}
\footnotetext{https://www.circle.com/en/usdc\#transparency}
\addtocounter{footnote}{1}
\footnotetext{https://paxos.com/busd-transparency/}
\addtocounter{footnote}{1}
\footnotetext{https://github.com/makerdao}


Stablecoins with the largest market capitalization (like USDT / USDC / BUSD / USDP and others) are 
backed by dollar-evaluated collateral with a 1-to-1 ratio. Issuers of such stablecoins 
provide documents that are issued by trusted authorities (auditors). Nevertheless, we consider 
these proofs as paper-proof that contains risks due to their centralized nature, which could 
be affected by corruption, human mistakes, and political influence.

At the moment of writing, the DAI stablecoin is the largest, most secure, and widespread 
decentralized stablecoin in the market. DAI's collateral (crypto assets) is evaluated at a 
147\% ratio to its market capitalization. Nevertheless, DAI stablecoin collateral has some 
notable nuances that bring risks to it (and some of them have already been realized):
\begin{itemize}
  \item DAI's collateral contains a major share of USDC stablecoin (74\% of DAI's capitalization 
  in April 2023), which is a cash/US treasuries-backed coin and is regulated by US financial authorities. 
  This means collateral risks of the USDC are also valid for DAI (in March of 2023, DAI depegged 
  from USD because USDC depegged too due to a bank issue, luckily the situation recovered later).
  \item USDC (a part of DAI's collateral) and some other stablecoins have an option of freezing 
  funds (which makes them insecure).
  \item Both DAI and USDC have legal entities in the US and are obliged to comply with all 
  regulator requirements, making them susceptible to influence.
  \item DAI contains a lending/borrowing protocol as a business model to support its existence, 
  which makes the whole system more complicated, fragile, and dependent on the results of this 
  business operating.
\end{itemize}

\subsection{Timing}

The period from 2022 to 2023 is regarded as a "crypto-fall" or "crypto-winter" time, during 
which the valuations of most crypto assets are low. Hopefully, this period will end soon, 
and the entire crypto market capitalization will increase by 3-5 times or more, as it has 
happened several times before. To achieve a high overcollateralization ratio, the USSD 
stablecoin was created during this time.

