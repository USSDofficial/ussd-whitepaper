
\section{Collateral structure}

The ideal crypto collateral structure should be robust enough to maintain a collateral factor 
of over 100\% even in harsh scenarios, such as an overall crypto market downturn, a 70-90\% 
depreciation of BTC/ETH, or even the complete depreciation of one of the crypto assets included 
as collateral. Due to the high volatility nature of crypto, we have decided to form a collateral 
structure that will:

\begin{enumerate}
  \item Diversify using the most common and trusted crypto assets, namely BTC and ETH.
  \item Include DAI as part of the collateral to enhance usability and liquidity in USSD stablecoin. 
  DAI was selected as the safe and decentralized stablecoin in the market at the time of USSD creation.
  \item Use the most decentralized blockchain that supports smart-contracts, namely the Ethereum Mainnet.
  \item Include a radical element in the collateral structure, namely the BGL (Bitgesell) coin, 
  which is deflationary and has a small capitalization (at the time of writing). It was included 
  in the collateral to overcome collateral fluctuations in the first period of USSD adoption and 
  to increase collateral upside in the long run.
  \item Have a flexible structure that changes with time and the market capitalization of the 
  collateral. The overall goal of the model is to accumulate collateral amounts that are at least 100\% 
  of the DAI and more than 10 times the crypto assets (BTC and ETH).
\end{enumerate}

Overview of the last two collateral component attributes.

BGL (Bitgesell) is a well-developed fork of Bitcoin with many parameters that 
remain the same, such as limited supply (21 million), but with the additional 
mechanics of burning 90\% of the transaction fees and having block reward halving 
every year. It is running on its own blockchain, which is very similar to Bitcoin. 
The reasons for selecting BGL as a part of collateral are:
\begin{itemize}
  \item Based on a proven code with similar blockchain safety as Bitcoin
  \item More coin scarcity in the long run compared to Bitcoin
  \item Small market cap that has significant upside potential
\end{itemize}
To grow and fix the share of DAI in the collateral structure, we have developed a roadmap 
for USSD collateral development (see Table 2). Once the share is fixed, the collateral structure 
will be locked forever to prevent any changes.


\begin{table}
\caption{Collateral buying proportions}
\centering
\begin{tabular}{|>{\hspace{0pt}}m{0.069\linewidth}|>{\hspace{0pt}}m{0.202\linewidth}|>{\hspace{0pt}}m{0.086\linewidth}|>{\hspace{0pt}}m{0.110\linewidth}|>{\hspace{0pt}}m{0.09\linewidth}|>{\hspace{0pt}}m{0.103\linewidth}|>{\hspace{0pt}}m{0.128\linewidth}|} 
\hline
\par{} & \makecell[l]{Collateral\\ buy proportion} & First flutter & Second flutter & Third flutter                          & Fourth flutter & Ultimate mode  \\ 
\hline
DAI    & 25\%                          & 0.25x         & 0.35x          & {\cellcolor[rgb]{0.737,0.941,0.855}}1x & 0.8x           & 0.25x          \\ 
\hline
wETH   & 25\%                          & 2x            & 4x             & 5x                                     & 6x             & More 10x       \\ 
\hline
wBTC   & 25\%                          & 2x            & 4x             & 5x                                     & 6x             & More 10x       \\ 
\hline
wBGL   & 25\%                          & 10x           & 20x            & 50x                                    & 100x           & More 500x      \\
\hline
\end{tabular}
\end{table}


In the event of a decline in the USSD price, the algorithm for selling collateral to maintain 
the peg at 1 USD is as follows: DAI will be sold first, followed by wETH, then wBTC, and finally, wBGL.

