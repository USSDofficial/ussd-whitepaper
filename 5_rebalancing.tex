\section{Collateral rebalancing mechanics}

The rebalance function is public and could be executed by anyone. This decision was 
made deliberately – providing less dependency on a single entity or project team 
supporting the calls for rebalancing.

In addition, rebalance also could be triggered before token transfer. If a price is 
exceeding some predefined threshold, the USSD would rebalance itself before any other 
transfer is done, providing it has an edge over other participants (who can try to 
front-run rebalancing).

Rebalancing is selling collateral and burning USSD for peg down recovery (supply 
contraction) and buying collateral components while minting USSD for peg up recovery 
(supply expansion).

The collateral has selling priority (order), which is:
1. DAI, 2. WETH, 3. WBTC, 4. WBGL. 

The goal of having an order of collateral liquidation is to have more volatile assets 
be further in the queue, allowing them to be accumulated and grow in value, while keeping 
DAI as near-line reserves and serve as a protection buffer from speculative manipulation.

When expanding its supply, USSD would buy assets in non-linear fashion, the main metrics 
defining this would be total supply and collateralization factor of collateral components. 
If total supply is less than 50000 USSD, the collateral is DAI-only. After initial growth, 
the collateral buying is trying to achieve a collateral proportion defined as tiers (or 
'flutters'), that are defined in table 2.

The current flutter level is determined as total collateralization level of USSD, e.g. 
first flutter is 0.25x + 2x + 2x + 10x = 14.25x. If collateralization factor is higher 
this value, then proportion is switched to the next flutter.

E.g., consider the following state of USSD and its collateral:
USSD total supply: 750000

\begin{table}
\caption{Collateral state example}
\centering
\begin{tabular}{|>{\hspace{0pt}}m{0.336\linewidth}|>{\hspace{0pt}}m{0.22\linewidth}|>{\hspace{0pt}}m{0.323\linewidth}|} 
\hline
Collateral stored\par{}(price measured in DAI) & Collateral factor & Target collateral factor               \\ 
\hline
289000 DAI                                     & 0.385             & 0.25                                   \\ 
\hline
125600 WETH                                    & 0.167             & 2                                      \\ 
\hline
450000 WBTC                                    & 0.6               & 2                                      \\ 
\hline
500000 WBGL                                    & 0.66              & 10                                     \\ 
\hline
Total:  1364600/750000                         & 1.812             & 14.25 (1\textsuperscript{st} flutter)  \\
\hline
\end{tabular}
\end{table}


The collateral is bought in equal proportions, if the collateral factor is lower than 
the target collateral factor. In the example above, DAI would not be bought, as it’s 
collateral factor of 0.385 is larger than target collateral factor of 0.25, and 1/3rd 
of the available buying power should be spent
to buy in equal proportions WETH, WBTC, WBGL (if the threshold of target collateral 
factor won’t be reached, otherwise the appropriate buying amount for collateral component 
would be lower). As swapping and buying requires gas, if a portion to buy a collateral 
component is less than 5\%, it is not bought (most probably it would be bought in 
the next operations).

Such mechanics also implies applying a cost-averaging accumulation of collateral assets. 
If a single asset depreciates, it's individual collateral factor decreases and amount 
to be bought increases.

Even if the collateral portion is higher than target collateral factor, no selling 
operations are performed during rebalancing.

Rebalancing is performed only if the relative proportion of tokens on the key USSD/DAI pool is
skewed larger than a certain threshold value (1\%), which is directly impacting swap price on
the pool, inside a rebalance function there is no external calls (outside Uniswap's contracts
and DAI transfer), and it's checked that amounts that are put into USSD/DAI pool make propotion
closer to 1-to-1. Even if some collateral pricing is a bit mismatched (e.g. Uniswap V3 on-chain
price oracle changes price once per block) the DAI that was accrued selling other collateral
would stay as DAI collateral on the USSD contract.

The rebalancing amount is also multiplied by a `safety buffer' (a value of 0.99), to take pool
fees into account and to ensure that rebalance operations don't cross 1.0 USSD/DAI boundary
to minimize arbitrage opportunities using USSD contract.

\subsection{Collateral storage}

The USSD contract acts itself as collateral holder. No other entity has access to USSD 
collateral except the contract's logic of collateral management, and DAI is also 
transferred during mint/redeem operations. In addition, at any moment, the collateral 
balances on the address of the USSD smart-contract serve as the proof of collateral 
presence and sufficiency. Collateral is not allocated in any other project 
(lending/borrowing platforms, investment vaults etc.) and is fully visible at all times.

USSD contract would provide and maintain allowances for itself, and collateral components
(DAI, WETH, WBTC, WBGL) to Uniswap V3 router for swap operations.

Prices of collateral components are measured through a series of contracts implementing simple
oracle-like interface. There's some flexibility reserved at the time of writing related to
the oracle implementations: there are numerous options available, such as using Chainlink
public feeds (available for larger assets such as DAI, WBTC, WETH), measure TWAP price
from Uniswap V3 pools, or other implementations. Oracle contract is assumed to return the price
in USD (but that is just the naming of the function), in theory, it could be returning price
in DAI. Various approaches to implementing price measurement have their own benefits and risks,
and could be a part of further research.



