\section{Introduction}

\subsection{Concept of Autonomy and Safety}

To make a significant contribution to the crypto economy's freedom 
and independence, a new secure and autonomous stablecoin is required 
to minimize the risks associated with centralization and crypto 
market volatility.

Existing stablecoins are susceptible to the following risks due to their 
strong connection to the traditional financial system, which is unpredictable:

\begin{enumerate}
\item Risks due to centralization:
	\begin{enumerate}
		\item Stable tokens can be banned/frozen, or the address can be blacklisted by authorized actors.
    \item The creators of stablecoins can be influenced using legal or other means.
    \item Some functions of stablecoin tokens might stop working if the project
    team stops supporting the project, such as collateral re-balancing or 
    mint/redeem for centralized stables.
    \item Collateral containing cash/treasuries or other stablecoins that include 
    cash/treasuries can collapse due to various situations in the banking system 
    or financial regulation issues.
  \end{enumerate}
\item Risks due to crypto market unpredictability and volatility:
	\begin{enumerate}
    \item A stablecoin might collapse if its internal business model operates incorrectly. 
    Dependency on the internal business model makes its behavior much more unpredictable, for example:
    \begin{itemize}
      \item For continuous support, a stablecoin project team needs funds. Thus, it needs 
      revenue generation in some form. For USDT/USDC and other stablecoins backed 
      by cash/treasures, it is the interest returned from deposits and treasury bond yields. 
      For DAI and other stablecoins backed by crypto assets, the revenue is fees and commissions 
      from lending and borrowing.
    \end{itemize}
    \item Collateral containing crypto assets can depreciate in value due to a decline 
    in the crypto market. If the crypto market is down significantly, it can unpeg 
    a crypto-backed stablecoin from USD.
  \end{enumerate}
\end{enumerate}

\begin{table}[]
\begin{tabular}{lllllll}
                                         & USDT                                     & USDC                                      & BUSD (USDP)                               & DAI                            & USSD                          &  \\
Transparency                             & Transparency based on 			auditor report1 & Transparent by paper proof2               & Transparent by paper proof 3              & Transparent by code4           & Transparent by code           &  \\
Source of Collateral                     & Short-term US Treasuries, cash, etc.     & Short-term US Treasures, cash in US banks & Short-term US Treasures, cash in US banks & Crypto backed (including USDC) & Crypto backed (including DAI) &  \\
Collateral ration                        & Claimed 100\%                            & 100\%                                     & 100\%                                     & $\sim$147\%                    & $\sim$500\%+                  &  \\
Regulated status                         & Hong Kong entity                         & US entity                                 & Several entities                          & US entity                      & No entity                     &  \\
Centralization level of token management & Team can froze tokens                    & Team can froze tokens                     & Team can froze tokens                     & Team can froze tokens          & No one can froze tokens       &  \\
Project Management                       & Centralized, manual                      & Centralized, manual                       & Centralized, manual                       & DAO managed, manual            & Autonomous                    &  \\
Business model                           & Treasures \%                             & Treasures \%                              & Treasures \%                              & Collateral service \%          & Non-profit                    & 
\end{tabular}
\end{table}

The USSD stablecoin aims to avoid the risks associated with existing stablecoins 
by implementing the following features:
\begin{enumerate}
  \item No methods for freezing, blacklisting, banning or pausing transfers are 
  implemented in the smart-contract. The goal is to make the USSD stablecoin completely unstoppable.
  \item USSD is a non-profit, long-term project that exists autonomously without any 
  connection to physical entities. Only code is used, making it non-biased and secure.
  \item The architecture of the USSD stablecoin is designed to be autonomous. In the 
  long-term, the operation of the USSD stablecoin would not be dependent on any person.
  \item Since the USSD protocol operates independently of any business model, it 
  eliminates the risks associated with the bankruptcy of such models.
  \item USSD contains only cryptoassets with no connection to cash in banks or any 
  other traditional financial instruments in the physical world, even low-risk ones.
  \item USSD's collateral structure is designed to maintain a collateral to capitalization 
  ratio of more than 10x.
  \item The USSD smart contract is deployed on the Ethereum main network because it is 
  the most common and trusted decentralized network that allows the building of a smart contract.
\end{enumerate}

Stablecoins with the largest market capitalization (like USDT/USDC/BUSD/USDP and others) are 
backed by dollar-evaluated collateral with a 1-to-1 ratio. Issuers of such stablecoins 
provide documents that are issued by trusted authorities (auditors). Nevertheless, we consider 
these proofs as paper-proof that contains risks due to their centralized nature, which could 
be affected by corruption, human mistakes, and political influence.

At the moment of writing, the DAI stablecoin is the largest, most secure, and widespread 
decentralized stablecoin in the market. DAI's collateral (crypto assets) is evaluated at a 
147\% ratio to its market capitalization. Nevertheless, DAI stablecoin collateral has some 
notable nuances that bring risks to it (and some of them have already been realized):
\begin{itemize}
  \item DAI's collateral contains a major share of USDC stablecoin (74\% of DAI's capitalization 
  in April 2023), which is a cash/US treasuries-backed coin and is regulated by US financial authorities. 
  This means collateral risks of the USDC are also valid for DAI (in March of 2023, DAI depegged 
  from USD because USDC depegged too due to a bank issue, luckily the situation recovered later).
  \item USDC (a part of DAI's collateral) and some other stablecoins have an option of freezing 
  funds (which makes them insecure).
  \item Both DAI and USDC have legal entities in the US and are obliged to comply with all 
  regulator requirements, making them susceptible to influence.
  \item DAI contains a lending/borrowing protocol as a business model to support its existence, 
  which makes the whole system more complicated, fragile, and dependent on the results of this 
  business operating.
\end{itemize}

\subsection{Timing}

The period from 2022 to 2023 is regarded as a "crypto-fall" or "crypto-winter" time, during 
which the valuations of most crypto assets are low. Hopefully, this period will end soon, 
and the entire crypto market capitalization will increase by 3-5 times or more, as it has 
happened several times before. To achieve a high overcollateralization ratio, the USSD 
stablecoin was created during this time.

\subsection{Collateral structure}

The ideal crypto collateral structure should be robust enough to maintain a collateral factor 
of over 100\% even in harsh scenarios, such as an overall crypto market downturn, a 70-90\% 
depreciation of BTC/ETH, or even the complete depreciation of one of the crypto assets included 
as collateral. Due to the high volatility nature of crypto, we have decided to form a collateral 
structure that will:

\begin{enumerate}
  \item Diversify using the most common and trusted crypto assets, namely BTC and ETH.
  \item Include DAI as part of the collateral to enhance usability and liquidity in USSD stablecoin. 
  Users will be able to redeem DAI in exchange for USSD at any time. DAI was selected as the safe 
  and decentralized stablecoin in the market at the time of USSD creation.
  \item Use the most decentralized blockchain that supports smart-contracts, namely the Ethereum Mainnet.
  \item Include a radical element in the collateral structure, namely the BGL (Bitgesell) coin, 
  which is deflationary and has a small capitalization (at the time of writing). It was included 
  in the collateral to overcome collateral fluctuations in the first period of USSD adoption and 
  to increase collateral upside in the long run.
  \item Have a flexible structure that changes with time and the market capitalization of the 
  collateral. The overall goal of the model is to accumulate collateral amounts that are at least 100\% 
  of the DAI and more than 10 times the crypto assets (BTC and ETH).
\end{enumerate}

Overview of the last two collateral component attributes.

BGL (Bitgesell) is a well-developed fork of Bitcoin with many parameters that 
remain the same, such as limited supply (21 million), but with the additional 
mechanics of burning 90\% of the transaction fees and having block reward halving 
every year. It is running on its own blockchain, which is very similar to Bitcoin. 
The reasons for selecting BGL as a part of collateral are:
\begin{itemize}
  \item Based on a proven code with similar blockchain safety as Bitcoin
  \item More coin scarcity in the long run compared to Bitcoin
  \item Small market cap that has significant upside potential
\end{itemize}
To grow and fix the share of DAI in the collateral structure, we have developed a roadmap 
for USSD collateral development (see Table 2). Once the share is fixed, the collateral structure 
will be locked forever to prevent any changes.


\begin{table}[]
\begin{tabular}{lllllll}
     & Distribution additional funds & First flutter & Second flutter & Third flutter       & Fourth flutter & Ultimate mode \\
DAI  & 25\%                          & 0.25x         & 0.35x          & 1x & 0.8x           & 0.25x         \\
wETH & 25\%                          & 2x            & 4x             & 5x                  & 6x             & More 10x      \\
wBTC & 25\%                          & 2x            & 4x             & 5x                  & 6x             & More 10x      \\
wBGL & 25\%                          & 10x           & 20x            & 50x                 & 100x           & More 500x    
\end{tabular}
\end{table}

In the event of a decline in the USSD price, the algorithm for selling collateral to maintain 
the peg at 1 USD is as follows: DAI will be sold first, followed by wETH, then wBTC, and finally, wBGL.


\section{USSD Supply management}

USSD coin is 'initialized' with one-time minting of 1000 USSD, taking 1000 DAI as collateral. 
This 1000 USSD would allow to create a pool USSD/DAI to have initial liquidity. Uniswap V3 
is selected as DEX of choice on Ethereum for being largest in volume and long-time reliable 
solution for on-chain swaps. As the pair is consisting of two stablecoins, the minimal fee 
tier (0.05\%) is used. Initial liquidity equation at the whole price range would be:

\[ {P_{USSD} \cdot P_{DAI} = 1}, \]

assuming equal price of USSD and DAI as a regular AMM curve that should fit the purpose 
of price discovery. Other curve variants, such as constant price () or Stableswap invariant 
(https://classic.curve.fi/files/stableswap-paper.pdf) serve different purposes (e.g. to 
prevent slippage when having a pool of stablecoins, even with significant different in 
their amounts).

The pool is used for price discovery of USSD, avoiding the necessity to use any off-chain\
oracles to keep USSD as autonomous as possible. For protection against flash-loan or 
liquidity manipulation geometric mean price oracle from Uniswap V3 is used 
(https://uniswap.org/whitepaper-v3.pdf):

\[ P_{ t1,t2 } = 1.0001^{ { a_{t2}-a_{t1} } \over {t_2 - t_1} }, \]

where t2 and t1 are define interval in seconds, and at1, at2 define accumulator values. 
The length of the interval could be fine-tuned, making the assumed price less susceptible 
for short-term fluctuations (and liquidity/loan attacks), gaining some latency, but spending 
less gas overall on rebalancing operations.

The main way to obtain the USSD would be buying it on the USSD/DAI pool, this would increase 
the price of USSD in comparison to DAI. In such scenario, two outcomes are possible: 
the price is arbitraged back (and nothing changes to the supply) or USSD rebalances itself, 
minting USSD and selling it for DAI on the pool, expanding the total supply of USSD.

In the opposite scenario (price of USSD goes below 1 DAI), USSD contract would reduce supply, 
using stored collateral to buy USSD back and burn it, reducing total supply of USSD and 
bringing the peg back to 1 DAI.

\section{Mint and redeem}

If there is positive DAI balance in the collateral, USSD contract can provide DAI for equal 
amount of USSD in return (that would be burned, contracting supply).

At any given time, USSD can be minted using DAI as collateral at 1-to-1 ratio, expanding 
total USSD supply.

Ability to mint and redeem USSD for DAI could serve as incentives to rebalance the coin 
when this is economically viable (covering the gas expenses). However, providing ability 
to mint or redeem USSD for collateral other than DAI (to which it aims to be pegged) 
could be exploited as getting a directional exposure to the collateral component 
(e.g. mint for WETH and then redeem for WETH later would lead to profiting from short 
exposure if WETH price decreases).

These methods also could be used to help USSD recover in negative scenarios: if USSD value 
falls below 1 DAI and there are less than 1 DAI reserves per USSD to refill the reserves 
allowing the USSD to recover it's price by reducing supply (at the expense of the agent 
performing that).

\section{Collateral rebalancing mechanics}

The rebalance function is public and could be executed by anyone. This decision was 
made deliberately – providing less dependency on a single entity or project team 
supporting the calls for rebalancing.

In addition, rebalance also could be triggered before token transfer. If a price is 
exceeding some predefined threshold, the USSD would rebalance itself before any other 
transfer is done, providing it has an edge over other participants (who can try to 
front-run rebalancing).

Rebalancing is selling collateral and burning USSD for peg down recovery (supply 
contraction) and buying collateral components while minting USSD for peg up recovery 
(supply expansion).

The collateral has selling priority (order), which is:
1. DAI, 2. WETH, 3. WBTC, 4. WBGL. 

The goal of having an order of collateral liquidation is to have more volatile assets 
be further in the queue, allowing them to be accumulated and grow in value, while keeping 
DAI as near-line reserves and serve as a protection buffer from speculative manipulation.

When expanding its supply, USSD would buy assets in non-linear fashion, the main metrics 
defining this would be total supply and collateralization factor of collateral components. 
If total supply is less than 50000 USSD, the collateral is DAI-only. After initial growth, 
the collateral buying is trying to achieve a collateral proportion defined as tiers (or 
'flutters'), that are defined in table 2.

The current flutter level is determined as total collateralization level of USSD, e.g. 
first flutter is 0.25x + 2x + 2x + 10x = 14.25x. If collateralization factor is higher 
this value, then proportion is switched to the next flutter.

E.g., consider the following state of USSD and its collateral:
USSD total supply: 750000

\begin{table}[]
\begin{tabular}{lllllll}
\begin{tabular}[c]{@{}l@{}}Collateral stored\\  (price measured in DAI)\end{tabular} & Collateral factor & Target collateral factor &  &  &  &  \\
289000 DAI                                                                           & 0.385             & 0.25                     &  &  &  &  \\
125600 WETH                                                                          & 0.167             & 2                        &  &  &  &  \\
450000 WBTC                                                                          & 0.6               & 2                        &  &  &  &  \\
500000 WBGL                                                                          & 0.66              & 10                       &  &  &  &  \\
Total:  1364600/750000                                                               & 1.812             & 14.25 (1st flutter)      &  &  &  &  \\
                                                                                     &                   &                          &  &  &  &  \\
                                                                                     &                   &                          &  &  &  & 
\end{tabular}
\end{table}

The collateral is bought in equal proportions, if the collateral factor is lower than 
the target collateral factor. In the example above, DAI would not be bought, as it’s 
collateral factor of 0.385 is larger than target collateral factor of 0.25, and 1/3rd 
of the available buying power should be spent
to buy in equal proportions WETH, WBTC, WBGL (if the threshold of target collateral 
factor won’t be reached, otherwise the appropriate buying amount for collateral component 
would be lower). As swapping and buying requires gas, if a portion to buy a collateral 
component is less than 5\%, it is not bought (most probably it would be bought in 
the next operations).

Such mechanics also implies applying a cost-averaging accumulation of collateral assets. 
If a single asset depreciates, it's individual collateral factor decreases and amount 
to be bought increases.

Even if the collateral portion is higher than target collateral factor, no selling 
operations are performed during rebalancing.

\subsection{Collateral storage}

The USSD contract acts itself as collateral holder. No other entity has access to USSD 
collateral except the contract's logic of collateral management, and DAI is also 
transferred during mint/redeem operations. In addition, at any moment, the collateral 
balances on the address of the USSD smart-contract serve as the proof of collateral 
presence and sufficiency. Collateral is not allocated in any other project 
(lending/borrowing platforms, investment vaults etc.) and is fully visible at all times.

USSD contract would provide and maintain allowances for itself, and collateral components
(DAI, WETH, WBTC, WBGL) to Uniswap V3 router for swap operations.

Prices of collateral components are measured in DAI (as USSD itself is pegged to DAI).

\subsection{Dependencies}

This is a list of dependencies the USSD smart contract has for its implementation
and functioning:

\begin{enumerate}
  \item OpenZeppelin contracts-upgradeable library
  https://github.com/OpenZeppelin/openzeppelin-contracts-upgradeable
  standard audited and widely used implementation of various useful libraries for 
  access management, transparent proxy pattern and others. To reserve ability to 
  patch possible bugs, the contract would be kept upgradeable (for a limited time, 
  and then locked afterwards).
  \item DAI stablecoin contract
       https://etherscan.io/token/0x6b175474e89094c44da98b954eedeac495271d0f
       DAI is used as metric of price and as part of the collateral.
  \item Uniswap V3 (router and pool contracts)
       Uniswap is used extensively by USSD to measure peg, collateral ratio, 
       perform collateral buying and selling.
\end{enumerate}





